\chapter{Introduction}
%Impact paragraph
Raising the awareness of developers to quickly update their third-party dependent library components (i.e., package dependencies) is now regarded seen as priority by both research and industry\cite{Bogart:2016, Mirhosseini:2017, Kula:2018, Kikas:2017}.
Once a newer version of a dependency is available, developers of the \textit{client project} (i.e., a project or package that uses the dependency) are strongly recommended to update their dependency.

%Vulnerabilities paragraph
As well as fixing bugs and adding new features, migration to a new version (i.e., updates) sometimes include fixes to prevent threats of unwanted access and potential malicious intent.
Such threats are regarded as a \textit{vulnerable dependency}.
To spread the awareness, vulnerable dependencies are typically assigned a Common Weaknesses and Exposures (CWE)\footnote{website at \url{https://cwe.mitre.org/}},  or Common Vulnerabilities and Exposures (CVE)\footnote{website at \url{https://cve.mitre.org/}}, then  archived\footnote{Vulnerabilities can be listed in common locations such as  NVD at (\url{https://nvd.nist.gov/}) or at CWE Details at (\url{https://www.cvedetails.com/})}. 

Prior empirical studies at the component level state that developers are slow to update their dependencies. 
These studies suggest that developers are driven by several factors, one being the amount of migration effort (i.e., modifying client code to integrate the newer version) needed to update.
However, a key threat to validity is that it was performed at the component rather than at the source code level. 

%Important related work
%Recently, researchers have empirically shown that client developers are not practicing library migrations \cite{Kula:2015, Decan:2018}, highlighting the perils and effect such as technical lag on the project, with rippling effect to the ecosystem \cite{Gonzalez-Barahona:2017}.
%Furthermore, studies have also shown that client developers struggle keeping up with updates, stating that either they were unaware of the opportunity, or that the cost benefit to update was a demotivating factor \cite{Bogart:2016}.

%We have mentioned before that even if there are upgrades available, developers will not update having one of the factors been the amount of migration effort. 
\section{Thesis Goals And Organization}
The goal of this thesis is to help the developers have a better understanding of the actual migration effort by providing detailed information of the projects and its dependencies.
First I will be discussing the background (Chapter \ref{sec:background}) which includes the motivating example. 
Next I will introduce related work that included tool support for library updates and prior work related to library dependencies and evolution (Chapter \ref{sec:related_work}).


The main contributions of this thesis are divided into three main sections
\begin{enumerate}
    \item \textbf{Exploratory Study of Vulnerable Function Call-Tracing:} I performed an exploratory study with the goal to understand to what extent the usage of library vulnerable code affects the way developers update their vulnerable dependencies. 
    I manually identified and validated vulnerability fixes and how they affected the client code. 
    In an empirical study of npm projects and their dependencies, I manually examined a total of 60 projects, investigating three cases of high priority vulnerabilities to understand how projects that use the vulnerable code and those that do not use it handle migration to fixed dependency versions. 
    Results of the exploratory study suggest that up to 73.3\% of the sampled outdated clients were indeed not reaching the vulnerability threat (i.e., did not execute the vulnerable code). 
    Furthermore, the study highlights how mapping vulnerable code to client usage is not trivial for JavaScript, with dependency information needed to understand the control flow and API detection of function difficulty. (Chapter \ref{sec:exploratory_study}).
%Analyzing a set of nodeJS projects to understand how vulnerabilities in a third-party library affects the way developers update their library dependencies.
    \item \textbf{\tool[]: A Tool to Aid in Library Migrations:} Once I completed the exploratory study I realized that the amount of manual work required to perform the study was very extent. In order to optimize the efforts needed for the analysis my goal was to create a tool that will be able to obtain the same information in a semi-automated way and at the same time allows the developers to use it to get information regarding their own projects. 
    To achieve this goal \tool[] was created,
    a tool that, in its main functionality, will allow the users to get a trace of the function-calls to a third party library methods.
    (Chapter \ref{sec:specification})
%Details on how, based on the results from the exploratory study, the tool was developed and the replication of the \textit{Vulnerable Function Call Tracing Exploratory Study} using \sj~in order to measure its performance and compare results. 
    \item \textbf{Evaluation and Results} 
    To evaluate the efficiency of \tool[] and how it impacts previous studies I performed 2 validations. (i) I replicated the exploratory study form the previous section using the tool in some of the steps.
    (ii) I evaluated a percentage of the projects from a previous study on vulnerabilities in the npm package dependency ecosystem \cite{decan2018impact} to make a comparison between analysis at library level (i.e. their approach) and function level (i.e. my approach).
    The execution and results of both validations will be presented in Chapters \ref{sec:validations} and \ref{sec:results}.
\end{enumerate}

%One of the main features of \sj~that we can highlight is the detection of the functions of the third-party library that are been used in the project (i.e. function calls), since nodeJS is not a compiled language as Java, the detection of the function calls is not a task easy to perform.
%To our knowledge there are very few tools, not to say nonexistent, that address this issue. 
%Therefore, we wanted the detection of third-party function calls to be an important feature of our tool. 
%Considering the above we will be dividing this thesis in to three main sections:

Finally I will be giving a conclusion that presents a summary and recommendations that ends the entire thesis (Chapter \ref{sec:conclusions}).