\newcommand{\edoctitle}{Master's Thesis} % or Master's Thesis
\newcommand{\degree}{Master} % or MASTER
\newcommand{\major}{ENGINEERING} % or SCIENCE
\newcommand{\studentnumber}{MT1751132}  % student number (DD: doctor, MT: Master)
\newcommand{\etitle}{\tool[]: A Function Level Analysis Tool To Aid In Smoother NodeJS Library Migrations}     % title of the dissertation
\newcommand{\eauthor}{Rodrigo Elizalde Zapata} % Author's name
\newcommand{\edate}{March
    \space\number 15,\space \number 2019}
\newcommand{\ekeywords}{NodeJS, libraries migration, npm, third party libraries, vulnerabilities, functions, dependencies}
\newcommand{\tool}[1][]{S\={o}jiTantei}

\newcommand{\eabstract}{ % abstract
 Nowadays it has become a common practice for software projects to  adopt  third-party  libraries,  allowing  developers  full  access to functions  that  otherwise  will  take  time,  effort and resources  to  create. Regardless of the migration effort involved, developers are encouraged to maintain and update any outdated dependency, so as to remain safe from potential threats including vulnerabilities. However, in many cases, despite having an updated version available, developers are not updating their dependencies. 
 My exploratory study show evidence that 73\% of the clients that were labeled vulnerable, for using a vulnerable version of a library, were actually not accessing the vulnerable code in the dependency confirming that analysis, at \textit{library level}, is indeed an overestimation.
 With this results in mind, in this thesis I propose \tool[] as an aid for developers on the node.js package manager (npm) ecosystem community by providing detailed information about the projects and its relationship with their dependencies at a \textit{function level} to motivate developers towards libraries migration.
}
