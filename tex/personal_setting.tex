\newcommand{\edoctitle}{Master's Thesis} % or Master's Thesis
\newcommand{\degree}{Master} % or MASTER
\newcommand{\major}{ENGINEERING} % or SCIENCE
\newcommand{\studentnumber}{MT1751132}  % student number (DD: doctor, MT: Master)
\newcommand{\etitle}{\tool[]: A Function Level Analysis Tool To Aid In Smoother NodeJS Library Migrations}     % title of the dissertation
\newcommand{\eauthor}{Rodrigo Elizalde Zapata} % Author's name
\newcommand{\edate}{March
    \space\number 15,\space \number 2019}
\newcommand{\ekeywords}{NodeJS, libraries migration, npm, third party libraries, vulnerabilities, functions, dependencies}
\newcommand{\tool}[1][]{S\={o}jiTantei}

\newcommand{\eabstract}{ % abstract
 Nowadays it has become a common practice for software projects to  adopt  third-party  libraries,  allowing  developers  full  access to functions  that  otherwise  will  take  time,  effort and resources  to  create. Regardless of the migration effort involved, developers are encouraged to maintain and update any outdated dependency, so as to remain safe from potential threats including vulnerabilities. However, in many cases, despite having an updated version available, developers are not updating their dependencies. 
 The preliminary results from the three studies performed in this thesis suggest that, labeling a client as vulnerable for using a vulnerable version of a library is actually an overestimation. In order to label a client as vulnerable, analysis at function level is required. However, to achieve this analysis an exhaustive manual work is necessary.
 With this in mind, I propose \tool[] as an aid for developers on the node.js package manager (npm) ecosystem community by providing, among others, a semi-automated function-call detection feature that will help developers have a better understanding of their project and its relation with the third-party libraries aiming for motivating the developers towards keeping their dependencies up to date.
}
